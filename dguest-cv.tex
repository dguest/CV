%% start of file `template.tex'.
%% Copyright 2006-2015 Xavier Danaux (xdanaux@gmail.com).
%
% This work may be distributed and/or modified under the
% conditions of the LaTeX Project Public License version 1.3c,
% available at http://www.latex-project.org/lppl/.


\documentclass[11pt,a4paper,sans]{moderncv}        % possible options include font size ('10pt', '11pt' and '12pt'), paper size ('a4paper', 'letterpaper', 'a5paper', 'legalpaper', 'executivepaper' and 'landscape') and font family ('sans' and 'roman')

%% custom commands
\newcommand{\arxiv}[1]{\href{http://arxiv.org/abs/#1}{\texttt{arXiv:#1}}}

\newlength{\savedindent}

% moderncv themes
\moderncvstyle{classic}                             % style options are 'casual' (default), 'classic', 'banking', 'oldstyle' and 'fancy'
\moderncvcolor{black}                               % color options 'black', 'blue' (default), 'burgundy', 'green', 'grey', 'orange', 'purple' and 'red'
%\renewcommand{\familydefault}{\sfdefault}         % to set the default font; use '\sfdefault' for the default sans serif font, '\rmdefault' for the default roman one, or any tex font name
%\nopagenumbers{}                                  % uncomment to suppress automatic page numbering for CVs longer than one page

% character encoding
%\usepackage[utf8]{inputenc}                       % if you are not using xelatex ou lualatex, replace by the encoding you are using
%\usepackage{CJKutf8}                              % if you need to use CJK to typeset your resume in Chinese, Japanese or Korean

% adjust the page margins
\usepackage[scale=0.75]{geometry}
\usepackage{amssymb}
%\setlength{\hintscolumnwidth}{3cm}                % if you want to change the width of the column with the dates
%\setlength{\makecvheadnamewidth}{10cm}            % for the 'classic' style, if you want to force the width allocated to your name and avoid line breaks. be careful though, the length is normally calculated to avoid any overlap with your personal info; use this at your own typographical risks...

\usepackage{bm}

% personal data
\name{Dan}{Guest}
\title{Curriculum Vitae}                               % optional, remove / comment the line if not wanted
\address{6 All\'ee Charles Baudelaire, Apt. A24}{01630 St.\,Genis-Pouilly}{France}% optional, remove / comment the line if not wanted; the "postcode city" and "country" arguments can be omitted or provided empty
%% \phone[mobile]{+1~(234)~567~890} 
% optional, remove / comment the line if not wanted; the optional "type" of the phone can be "mobile" (default), "fixed" or "fax"
%% \phone[fixed]{+2~(345)~678~901}
%% \phone[fax]{+3~(456)~789~012}
\email{dguest@cern.ch}                               % optional, remove / comment the line if not wanted
%\homepage{www.johndoe.com}                         % optional, remove / comment the line if not wanted
%\social[linkedin]{john.doe}                        % optional, remove / comment the line if not wanted
%\social[xing]{john\_doe}                           % optional, remove / comment the line if not wanted
%\social[twitter]{jdoe}                             % optional, remove / comment the line if not wanted
\social[github]{dguest}                              % optional, remove / comment the line if not wanted
%\social[gitlab]{https://gitlab.cern.ch/dguest}                              % optional, remove / comment the line if not wanted
%\social[skype]{jdoe}                               % optional, remove / comment the line if not wanted
%\extrainfo{additional information}                 % optional, remove / comment the line if not wanted
%\quote{Some quote}                                 % optional, remove / comment the line if not wanted

% bibliography adjustements (only useful if you make citations in your resume, or print a list of publications using BibTeX)
%   to show numerical labels in the bibliography (default is to show no labels)
%\makeatletter\renewcommand*{\bibliographyitemlabel}{\@biblabel{\arabic{enumiv}}}\makeatother
\renewcommand*{\bibliographyitemlabel}{[\arabic{enumiv}]}
%   to redefine the bibliography heading string ("Publications")
%\renewcommand{\refname}{Articles}

% bibliography with mutiple entries
%\usepackage{multibib}
%\newcites{book,misc}{{Books},{Others}}
%----------------------------------------------------------------------------------
%            content
%----------------------------------------------------------------------------------
\begin{document}
%\begin{CJK*}{UTF8}{gbsn}                          % to typeset your resume in Chinese using CJK
%-----       resume       ---------------------------------------------------------
\makecvtitle

\section{Education}
\cventry{2016}{Ph.D. Physics}{Yale University}{New Haven, Connecticut, USA}{}{Thesis: A Search for Scalar Charm Quarks with the ATLAS Detector at the LHC}  % arguments 3 to 6 can be left empty
\cventry{2007}{B.A. Physics}{Amherst College}{Amherst, Massachusetts, USA}{\textit{magna cum laude}}{Thesis: A cross-Beam Far Off-Resonance Optical Trap for $^{87}\mathrm{Rb}$ Bose-Einstein Condensates}

%% \section{Master thesis}
%% \cvitem{title}{\emph{Title}}
%% \cvitem{supervisors}{Supervisors}
%% \cvitem{description}{Short thesis abstract}

\section{Employment}
\cventry{Jul 2020--}{Postdoctoral Researcher}{Humboldt University}{Berlin, DE}{}{Researcher on ATLAS Experiment}
\cventry{2015--2020}{Postdoctoral Researcher}{University of California, Irvine}{Irvine, CA, USA}{}{Researcher on ATLAS Experiment}

\section{Research Appointments in the ATLAS Experiment}

%% \cventry{2009--}{Physicist}{ATLAS Collaboration}{Geneva, Switzerland}{}{%
%%   \begin{itemize}%
%%   %% \item Searches for low-mass hadronic resonances
%%   %% \item Searches for dark matter produced in association with $W$, $Z$, or Higgs bosons
%%   \item Liaison with University of California, Irvine (UCI) Computer Science Department
%%   \item Software developer for FELIX component of the ATLAS Run 3 data acquisition system
%%   %% \item Searches for supersymmetric particles decaying to charm quarks
%%   \item Developer for $b$- and $c$-jet identification algorithms
%%   %% \item Efficiency and resolution for drift tubes in the ATLAS Transition Radiation Tracker
%%   \end{itemize}
%% }

\cventry{2023--2025}{Convener}{}{Flavor Tagging Group}{}{}
\cventry{2020--2022}{Signature Coordinator}{$b$-jets}{Trigger Group}{}{}
\cventry{2018--2021}{Analysis Contact}{Mono-$H$, $H \to b\bar{b}$}{Exotics Group}{}{}
\cventry{2018--2020}{Convener}{Machine Learning Forum}{Computing Group}{}{}
\cventry{2017--2020}{Flavor Tag Contact}{Boosted $H \to b\bar{b}$ Taskforce}{Atlas Combined Performance}{}{}
\cventry{Fall~2018}{Taskforce Chair}{Software and Database Review}{Flavor Tagging Group}{}{}

\section{Research Experience}
\cventry{2012--}{Algorithm Developer}{Flavor Tagging}{ATLAS Experiment}{}{}
\cventry{2016--2017}{Software Developer}{Front End Links (FELIX)}{Trigger and Data Acquisition}{}{}
\cventry{2009--2012}{Detector Studies}{Transition Radiation Tracker}{ATLAS Experiment}{}{}
\cventry{2005--2007}{Research Assistant}{Hall Labs}{Amherst MA, USA}{}{%
  \begin{itemize}
  \item Constructed and commissioned a far off-resonance optical trap for Bose-Einstein Condensates
  \item Optics fabrication with vacuum deposition
  \end{itemize}
}
\cventry{2003--2004}{Research Assistant}{Dartmouth Medical School}{Hanover NH, USA}{}{\begin{itemize}
  \item Gathering physiological data from lab animals
  \item Writing data analysis software
  \end{itemize}
}



%% \section{Research Interests}
%% \cvitemwithcomment{Language 1}{Skill level}{Comment}
%% \cvitemwithcomment{Language 2}{Skill level}{Comment}
%% \cvitemwithcomment{Language 3}{Skill level}{Comment}

%% \section{Computer skills}
%% \cvdoubleitem{category 1}{XXX, YYY, ZZZ}{category 4}{XXX, YYY, ZZZ}
%% \cvdoubleitem{category 2}{XXX, YYY, ZZZ}{category 5}{XXX, YYY, ZZZ}
%% \cvdoubleitem{category 3}{XXX, YYY, ZZZ}{category 6}{XXX, YYY, ZZZ}


%% \section{Extra 1}
%% \cvlistitem{Item 1}
%% \cvlistitem{Item 2}
%% \cvlistitem{Item 3. This item is particularly long and therefore normally spans over several lines. Did you notice the indentation when the line wraps?}

%% \section{Extra 2}
%% \cvlistdoubleitem{Item 1}{Item 4}
%% \cvlistdoubleitem{Item 2}{Item 5\cite{book1}}
%% \cvlistdoubleitem{Item 3}{Item 6. Like item 3 in the single column list before, this item is particularly long to wrap over several lines.}


\section{Publications}
\subsection{Hundreds of published papers as part of the ATLAS Collaboration with over 10,000 citations.}

\cvitem{June 2024}{The ATLAS Trigger System for LHC Run 3 and Trigger performance in 2022\newline{}
  \texttt{\href{https://doi.org/10.1088/1748-0221/19/06/P06029}{JINST 19 (2024) P06029}}, \arxiv{2401.06630}
}
\cvitem{July 2023}{ATLAS flavour-tagging algorithms for the LHC Run 2 $pp$ collision dataset\newline{}
\texttt{\href{https://doi.org/10.1140/epjc/s10052-023-11699-1}{Eur. Phys. J. C 83 (2023) 681}}, \arxiv{2211.16345}}
\cvitem{Nov 2023}{Fast $b$-tagging at the high-level trigger of the ATLAS experiment in LHC Run 3\newline{}
  \texttt{\href{https://doi.org/10.1088/1748-0221/18/11/P11006}{JINST 18 (2023) P11006}}, \arxiv{2306.09738}}
\cvitem{Nov 2021}{Search for dark matter produced in association with a Standard Model Higgs boson decaying into $b$-quarks using the full Run 2 dataset from the ATLAS detector\newline{}
  \texttt{\href{https://doi.org/10.1007/JHEP11(2021)209}{JHEP 11 (2021) 209}}, \arxiv{2108.13391}}
\cvitem{Oct 2019}{Identification of boosted Higgs bosons decaying into $b$-quark pairs with the ATLAS detector at 13 TeV\newline{}
\textit{\href{https://doi.org/10.1140/epjc/s10052-019-7335-x}{Eur. Phys. J. C 79 (2019) 836}}, \arxiv{1906.11005}}
\cvitem{Aug 2019}{Search for low-mass resonances decaying into two jets and produced in association with a photon using $pp$ collisions at $\sqrt{s}=13\,\mathrm{TeV}$ with the ATLAS detector\newline{}
\textit{\href{https://doi.org/10.1016/j.physletb.2019.03.067}{Phys. Lett. B 795 (2019) 56}}, \arxiv{1901.10917}}
%% \cvitem{Nov 2017}{Search for Dark Matter Produced in Association with a Higgs Boson Decaying to $b \bar{b}$ Using 36 $\mathrm{fb}^{-1}$ of $pp$ Collisions at $\sqrt{s}=13\,\mathrm{TeV}$ with the ATLAS Detector\newline{} \textit{\href{https://doi.org/10.1103/PhysRevLett.119.181804}{Phys. Rev. Lett. 119, 181804}}, \arxiv{1707.01302}}
%% \cvitem{July 2017}{Optimisation and performance studies of the ATLAS $b$-tagging algorithms for the 2017-18 LHC run\newline{}
%%   \textit{Available from CERN: }\url{https://cds.cern.ch/record/2273281}}
\cvitem{June 2017}{Variable Radius, Exclusive-$k_{\mathrm{T}}$, and Center-of-Mass Subjet Reconstruction for Higgs($\to b\bar{b}$) Tagging in ATLAS\newline{}
  \textit{Avaliable from CERN:} \url{https://cds.cern.ch/record/2268678}}
\cvitem{March 2017}{Identification of Jets Containing $b$-Hadrons with Recurrent Neural Networks at the ATLAS Experiment\newline{}
  \textit{Available from CERN:} \url{https://cds.cern.ch/record/2255226}}
\cvitem{Dec 2016}{Jet flavor classification in high-energy physics with deep neural networks\newline{} \textit{\href{https://doi.org/10.1103/PhysRevD.94.112002}{Phys. Rev. D 94, 112002}}, \arxiv{1607.08633}}
\cvitem{April 2015}{Search for Scalar Charm Quark Pair Production in $pp$ Collisions at $\sqrt{s}=8\,\mathrm{TeV}$ with the ATLAS Detector\newline{} \textit{\href{https://doi.org/10.1103/PhysRevLett.114.161801}{Phys. Rev. Lett. 114, 161801}}, \arxiv{1501.01325}}
\cvitem{Sept 2014}{Search for pair-produced third-generation squarks decaying via charm quarks or in compressed supersymmetric scenarios in $pp$ collisions at $\sqrt{s}=8\,\mathrm{TeV}$ with the ATLAS detector\newline{} \textit{\href{https://doi.org/10.1103/PhysRevD.90.052008}{Phys. Rev. D 90, 052008}}, \arxiv{1407.0608}}
\cvitem{Jan 2015}{Performance and Calibration of the JetFitterCharm Algorithm for $c$-Jet Identification\newline{}
  \textit{Available from CERN:} \url{https://cds.cern.ch/record/1980463}}



\section{Conferences, Seminars, and Tutorials {\small ($\bigstar$ event organizer)}}

\cventry{Nov 2022}{Probing the nature of electroweak symmetry breaking with Higgs boson pairs in ATLAS }{\href{https://indico.cern.ch/event/1151741/contributions/5047067/}{SILAFAE 2022}}{Quito, Ecuador}{}{}
\cventry{Nov 2019}{$\bigstar$ \href{https://indico.cern.ch/event/844092}{4th ATLAS Machine Learning Workshop}}{}{CERN}{}{}
\cventry{August 2019}{Machine Learning in Particle Experiments}{\href{https://indico.cern.ch/event/783977/contributions/3467062/}{Dark Matter @ LHC 2019}}{Seattle, Washington, USA}{}{}
\cventry{June 2019}{$\bigstar$ Machine Learning Tutorial }{\href{https://indico.cern.ch/event/795039/contributions/3455082/}{ATLAS Tracking and Flavour Tagging Workshop}}{DESY, Hamburg}{}{}
\cventry{March 2019}{Machine Learning in High Energy Physics}{\href{https://indico.cern.ch/event/748043/contributions/3326031/}{2019 Winter Conference: ``In Persuit of New Particles and Paradigms''}}{Aspen, Colorado, USA}{}{}
\cventry{Oct 2018}{$\bigstar$ \href{https://indico.cern.ch/event/735932/}{3rd ATLAS Machine Learning Workshop}}{}{CERN}{}{}
\cventry{Dec 2017}{Boosted Object Tagging: Flavor Tagging of Boosted Objects}{\href{https://indico.cern.ch/event/675559/}{ATLAS Workshop: Physics with $120\,\mathrm{fb}^{-1}$}}{CERN}{}{}
\cventry{Sept 2017}{Searches For SUSY and Dark Matter from ATLAS and CMS}{\href{https://indico.cern.ch/event/659310/}{Top 2017}}{Braga, Portugal}{}{}
\cventry{Sept 2017}{High-Level Tagging Tools: A 5 Year Plan}{\href{https://indico.cern.ch/event/631313/contributions/2700470/}{ATLAS Joint Flavor Tagging and $H \to bb$ Workshop}}{Stony Brook, New York, USA}{}{}
\cventry{August 2017}{Boosted Object Tagging: $\bm{H \to bb}$ Tagging}{\href{https://indico.cern.ch/event/642438/}{ATLAS Hadronic Calibration Workshop}}{Toronto, Canada}{}{}
\cventry{July 2017}{How do we define Jets?}{\href{https://www.weizmann.ac.il/conferences/SRitp/Summer2017/}{Hammers and Nails --- Machine Learning \& HEP}}{Weizmann Institute of Science}{Rehovot, Israel}{}{}
\cventry{June 2017}{Machine Learning for Object Identification at the LHC}{\href{https://www.uni-goettingen.de/en/seminars/496592.html}{Seminars of the 2nd Institute of Physics}}{G\"ottingen, Germany}{}{}
\cventry{May 2017}{Introduction to High Level Taggers}{\href{https://indico.cern.ch/event/615994/}{ATLAS Workshop on Machine Learning and $b$-tagging}}{SLAC, Menlo Park, California, USA}{}{}
\cventry{May 2017}{Jet $\bm{b}$-tagging with Recurrent Neural Networks}{\href{https://indico.fnal.gov/event/13497/}{DS@HEP 2017}}{Fermilab, Illinois, USA}{}{}
\cventry{July 2016}{Deep Learning for Boosted Object Tagging}{\href{https://indico.cern.ch/event/439039/}{Boost 2016}}{Zurich, Switzerland}{}{}
\cventry{Jan 2014}{Collider Cross-Talk: Charm tagging in ATLAS \& CMS}{The Top-Charm Frontier at the LHC}{CERN, Switzerland}{}{}
\cventry{Jan 2014}{Compressed Stop to Charm Searches in ATLAS}{The Top-Charm Frontier at the LHC}{CERN, Switzerland}{}{}
\cventry{Dec 2013}{Searches for Direct Pair Production of Third Generation Squarks with the ATLAS Detector}{\href{https://www.phy.ncu.edu.tw/hep/pascos2013/}{Pascos 2013}}{Taipei, Taiwan}{}{}

\subsection{As Co-author}

\cventry{May 2008}{Vortex Lattices in a Crossed-Beam Optical Dipole Trap}{39th Annual Meeting of the APS Division of Atomic, Molecular, and Optical Physics}{State Collage, Pennsylvania, USA}{}{}

\section{Other Research Interests}
%% \cvitem{Physics}{Focusing on LHC hadronic resonances and dark matter searches}
\cvitem{Computing}{Focused on deep neural networks and data pipelines}
\cvitem{Education}{Increasing knowledge transfer between high-energy physics and the broader data-science community}
%% \cvitem{Software Infrastructure}{Particularly streamlining the application of machine learning algorithms within performance-critical data pipelines}


\section{Teaching}

\cventry{August 2019}{ATLAS Software Bootcamp}{\href{https://smeehan12.github.io/2019-08-19-usatlas-computing-bootcamp/}{Lawrence Berkeley National Laboratory}}{}{}{Version control, containers, and continuous integration for ATLAS workflows}
\cventry{October 2014}{Software Carpentry}{Yale University}{}{}{Practical Bash, Python, and version control for physical sciences}
\cventry{Spring 2010}{Advanced Lab}{Yale University}{}{}{Teaching assistant for Steve Lamoreaux}
\cventry{Fall 2009}{Intermediate Lab}{Yale University}{}{}{Teaching assistant for Andreas Heinz}
\cventry{Spring 2009}{Advanced General Physics}{Yale University}{}{}{Teaching assistant for Sohrab Ismall-Beigi}
\cventry{Fall 2008}{General Physics Laboratory}{Yale University}{}{}{Teaching assistant for David DeMille}
\cventry{Spring 2008}{General Physics Laboratory}{Yale University}{}{}{Teaching assistant for David DeMille}
\cventry{2007-2008}{Teaching Fellow}{Amherst College}{}{}{Full-time teaching assistant for freshman-level physics}
\cventry{Fall 2005}{Methods of Theoretical Physics}{Amherst College}{}{}{Teaching assistant for Kannan Jagannathan}

\section{Mentoring}

\subsection{Graduate Students}
\cventry{2020--}{Victor Hugo Ruelas Rivera}{}{Humboldt University of Berlin}{}{
  \begin{itemize}
  \item Software and Machine Learning for the ATLAS High Level Trigger
  \item Probing di-Higgs coupling via $HH \to b\bar{b}b\bar{b}$
  \end{itemize}
}
\cventry{2016--2020}{Yvonne Ng}{}{UC Irvine (Physics)}{}{
  \begin{itemize}
  \item Gaussian Processes for background modeling
  \item Searches for low-mass hadronic resonances
  \end{itemize}
}
\cventry{2015--2020}{Julian Collado}{}{UC Irvine (Computer Science)}{}{Machine Learning for boosted Higgs tagging}
\cventry{2015--2020}{Daniel Antrim}{}{UC Irvine (Physics)}{}{
  General software and physics mentoring
}
\cventry{2016--2017}{Joshua Smith}{}{G\"ottingen (Physics)}{}{
  Machine learning for LHC data analysis}
\cventry{2015--2017}{Kevin Bauer}{}{UC Irvine (Physics)}{}{
  \begin{itemize}
  \item Search for $Z'$ production in association with Dark Matter
  \item Software development for ATLAS data acquisition system
  \end{itemize}
}
\subsection{Undergraduates}
\cventry{2013--2014}{Luke de Olivera}{}{Yale University}{}{Machine learning for $b$ and $c$-tagging}
\cventry{2013}{David Roeca}{}{Yale University}{}{Supersymmetric charm searches}
\cventry{2012}{Ariel Ekblaw}{}{Yale University}{}{Supersymmetric top decaying to charm jets}

%% \section{References}
%% \begin{cvcolumns}
%%   \cvcolumn{Category 1}{\begin{itemize}\item Person 1\item Person 2\item Person 3\end{itemize}}
%%   \cvcolumn{Category 2}{Amongst others:\begin{itemize}\item Person 1, and\item Person 2\end{itemize}(more upon request)}
%%   \cvcolumn[0.5]{All the rest \& some more}{\textit{That} person, and \textbf{those} also (all available upon request).}
%% \end{cvcolumns}

% Publications from a BibTeX file without multibib
%  for numerical labels: \renewcommand{\bibliographyitemlabel}{\@biblabel{\arabic{enumiv}}}% CONSIDER MERGING WITH PREAMBLE PART
%  to redefine the heading string ("Publications"): \renewcommand{\refname}{Articles}
\nocite{*}
\bibliographystyle{plain}
\bibliography{publications}                        % 'publications' is the name of a BibTeX file

% Publications from a BibTeX file using the multibib package
%\section{Publications}
%\nocitebook{book1,book2}
%\bibliographystylebook{plain}
%\bibliographybook{publications}                   % 'publications' is the name of a BibTeX file
%\nocitemisc{misc1,misc2,misc3}
%\bibliographystylemisc{plain}
%\bibliographymisc{publications}                   % 'publications' is the name of a BibTeX file

\end{document}

\clearpage
%-----       letter       ---------------------------------------------------------
% recipient data
\recipient{Company Recruitment team}{Company, Inc.\\123 somestreet\\some city}
\date{January 01, 1984}
\opening{Dear Sir or Madam,}
\closing{Yours faithfully,}
\enclosure[Attached]{curriculum vit\ae{}}          % use an optional argument to use a string other than "Enclosure", or redefine \enclname
\makelettertitle

Lorem ipsum dolor sit amet, consectetur adipiscing elit. Duis ullamcorper neque sit amet lectus facilisis sed luctus nisl iaculis. Vivamus at neque arcu, sed tempor quam. Curabitur pharetra tincidunt tincidunt. Morbi volutpat feugiat mauris, quis tempor neque vehicula volutpat. Duis tristique justo vel massa fermentum accumsan. Mauris ante elit, feugiat vestibulum tempor eget, eleifend ac ipsum. Donec scelerisque lobortis ipsum eu vestibulum. Pellentesque vel massa at felis accumsan rhoncus.

Suspendisse commodo, massa eu congue tincidunt, elit mauris pellentesque orci, cursus tempor odio nisl euismod augue. Aliquam adipiscing nibh ut odio sodales et pulvinar tortor laoreet. Mauris a accumsan ligula. Class aptent taciti sociosqu ad litora torquent per conubia nostra, per inceptos himenaeos. Suspendisse vulputate sem vehicula ipsum varius nec tempus dui dapibus. Phasellus et est urna, ut auctor erat. Sed tincidunt odio id odio aliquam mattis. Donec sapien nulla, feugiat eget adipiscing sit amet, lacinia ut dolor. Phasellus tincidunt, leo a fringilla consectetur, felis diam aliquam urna, vitae aliquet lectus orci nec velit. Vivamus dapibus varius blandit.

Duis sit amet magna ante, at sodales diam. Aenean consectetur porta risus et sagittis. Ut interdum, enim varius pellentesque tincidunt, magna libero sodales tortor, ut fermentum nunc metus a ante. Vivamus odio leo, tincidunt eu luctus ut, sollicitudin sit amet metus. Nunc sed orci lectus. Ut sodales magna sed velit volutpat sit amet pulvinar diam venenatis.

Albert Einstein discovered that $e=mc^2$ in 1905.

\[ e=\lim_{n \to \infty} \left(1+\frac{1}{n}\right)^n \]

\makeletterclosing

%\clearpage\end{CJK*}                              % if you are typesetting your resume in Chinese using CJK; the \clearpage is required for fancyhdr to work correctly with CJK, though it kills the page numbering by making \lastpage undefined
\end{document}


%% end of file `template.tex'.
